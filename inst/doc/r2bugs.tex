\documentclass[12pt]{article}
\usepackage{Sweave}
\usepackage{amsmath}
\usepackage{bm}
\usepackage[authoryear,round]{natbib}
\bibliographystyle{plainnat}
   
\title{Simulation and Parameter Estimation for Biomass Crops}
\author{David S. LeBauer, Ben Bolker}
\begin{document}
\Sconcordance{concordance:r2bugs.tex:r2bugs.Rnw:%
1 20 1 1 6 4 1 1 2 1 0 1 2 1 0 1 2 1 1 11 0 1 2 4 1 1 7 7 1}



\setkeys{Gin}{width=\textwidth}
\newcommand{\code}[1]{\texttt{\small{#1}}}

\maketitle
\begin{abstract}
Demonstration of the use of the \code{r2bugs.distributions} function.
\end{abstract}


\section{Introduction}

A simple example follows, using a normal distribution, $N(\mu = 10, \sigma = 2)$ we will compare a random sample from the R and BUGS implementation of the distribution.

\begin{Schunk}
\begin{Sinput}
> r.distn <- data.frame(distn = "norm", parama = 10, paramb = 2)
> Y.R <- do.call(paste("r", r.distn$distn, sep = ""), 
+                list(n.iter/4, r.distn$parama, r.distn$paramb))
> bugs.dist <- r2bugs.distributions(r.distn)
> Y.BUGS <- r2bugs::bugs.rdist(bugs.dist, n.iter = n.iter)
\end{Sinput}
\begin{Soutput}
Compiling model graph
   Resolving undeclared variables
   Allocating nodes
   Graph Size: 4

Initializing model
\end{Soutput}
\end{Schunk}


Simple tests show that the mean and variance of the two samples are similar. Indeed, this is done for five parameterizations of each of seven distributions (Normal, log-Normal, Weibull, Gamma, $\chi^2$, Binomial, and Negative-Binomial in the tests that are written for the \code{r2bugs} package. These tests and can be found in the file \code{/inst/tests/test.r2bugs.distributions.R}.

Here, we can visually compare the similarity of the density of 25000 samples from the same distribution implemented in R (black) and then JAGS (red).

\begin{figure}[htbp!]
  \centering
  \includegraphics[width=0.5\textwidth]{fig1}
\end{figure}

\end{document}

