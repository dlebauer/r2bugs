\title{Translating Probability Distributions:\\ From R to BUGS and Back Again}
\author{by David S. LeBauer, Michael C. Dietze, Ben M. Bolker}

\maketitle

\abstract{
The ability to implement statistical models in the BUGS language facilitates Bayesian inference by automating MCMC algorithms.
Software written in the BUGS language include OpenBUGS, WinBUGS, and JAGS.
A suite of R packages integrate this software seamlessly into the R environemnt for use pre-processing data and analyzing model output.

However, the R and BUGS languages use inconsistent representations of common probability distributions, and this creates the potential for error and confusion when using both languages. 
Here we review different parameterizations used by the R and BUGS languages, describe how to translate between the languages, and provide an R function, \code{r2bugs.distributions}, that transforms parameterizations from R to BUGS, and back again.
}

First, we summarize distributions for which the default parameterizations used by R and BUGS are different (Table~\ref{tab:parameterizations}).

\begin{table}[h]
\begin{center}
{\small
\begin{tabular}{llllll}
\hline
Distribution       &  Language  &  Parameterization  &  Use &  Notes \\
\hline
Normal&&&&\\
&  R      &  $\frac{1}{\sqrt{2 \pi}\sigma}\exp(-\frac{(x - \mu)^2}{2 \sigma^2})$ & \code{dnorm(x, mean = $\mu$, sd = $\sigma$)}& \\
&  BUGS   &  $\sqrt{\frac{\tau}{2\pi}}\exp(-(x-\mu)^2\tau)$ & \code{dnorm(mean = $\mu$, precision = $\tau$)} & $\tau=1/\sigma^2$  \\
log-Normal&&&&\\
&  R      &  $\frac{1}{\sqrt{2 \pi} \sigma x} \exp(-\frac{(\textrm{log}(x) - \mu)^2}{(2 \sigma^2)})$  &  \code{dlnorm(x, mean = $\mu$, sd = $\sigma$)}  & \\
&  BUGS   &  $\frac{\sqrt{\tau}}{x}\exp(\frac{-\tau(\textrm{log}(x)-\mu)^2}{2})$                     &  \code{dlnorm(mean = $\mu$, precision = $\tau$)} & $\tau=1/\sigma^2$\\
Binomial&&&&\\
&  R      &  ${n \choose x} p^{x}(1-p)^{n-x}$                                               &  \code{dbinom(x, size = $n$, prob = $p$)}  &  \\
&  BUGS   &  same   &  \code{dbin(prob = $p$, size = $n$)} & reverse parameter order \\
%
Negative Binomial&&&&reverse parameter order\\
&  R      &  $\frac{\Gamma(x+n)}{\Gamma(n) x!} p^n (1-p)^x$ &  \code{dnbinom(x, size = $n$, prob = $p$)} & size (\code{n}) is continuous    &     \\
%&  R$^*$      &  $\frac{\Gamma(k+x)}{\Gamma(k)x!}(\frac{k}{k+\mu})^k \frac{\mu}{(k+\mu)^x}$     &  \code{dnbinom(x, size = $n$, mu = $\mu$)} & \\%$\mu=n(1-p)/p$  \\
&  BUGS   &  ${x+r-1 \choose x}p^r(1-p)^x$ &  \code{dnegbin(prob = $p$, size = $r$)}     & size (\code{r}) is discrete  \\ 
Weibull&&&&\\
&  R      &  $\frac{a}{b} (\frac{x}{b})^{a-1} \exp(- (\frac{x}{b})^a)$                      &  \code{dweibull(x, shape = a, scale = b)}          & \\
&  BUGS   &  $\nu\lambda x^{\nu - 1}\exp(-\lambda x^{\nu})$  &  \code{dweib(shape = $\nu$, lambda = $\lambda$)}   & $\lambda=(1/b)^a$  \\
%%
Gamma&&&&reverse parameter order \\
&  R      &  ${\frac{r^a}{\Gamma(a)}} x^{a-1} \exp(-xr)$                                    &  \code{dgamma(x, shape = $$a, rate = $r$)} & \\
%&  R$^*$      &  ${\frac{1}{s^{a}\Gamma(a)}} x^{a-1} \exp(-x/s)$                                &  \code{dgamma(x, shape = a, scale = s)} &  \\
&  BUGS   &  ${\frac{\lambda^r x^{r-1}\exp(-\lambda x)}{\Gamma(r)}}$                        &  \code{dgamma(shape = $r$, lambda = $\lambda$)}  &  \\
\hline
\end{tabular}
}
\end{center}
\caption{\small{ Summary of different parameterizations of common distributions used by R and BUGS. The random variable $x$ is implicit in all of the BUGS ``Use'' parameterizations. For clarity and ease of reference, parameterizations follow the JAGS and R documentation; thus, the table includes some equivalent variables with different names (e.g. for Gamma,  $r$ in BUGS and $a$ in R are precisely the same), and equivalent expressions with different forms (which motivates this article). The order of parameters matters, since argument names are not used in BUGS and are optional in R.
This is especially important because the order of parameters for the \strong{Binomial} and \strong{Negative Binomial} distributions are switched. This table only includes R's default parameterizations; the \strong{Negative Binomial} and \strong{Gamma} distributions have alternative parameterizations if the arguments are named. The \strong{Beta}, \strong{Poisson}, \strong{Exponential}, and \strong{Uniform} distributions have the same parameterizations in both BUGS and R.}}
\label{tab:parameterizations}
\end{table}


To support the use of informative prior distributions in meta-analysis \citep{lebauer2012ffb}, we developed the \code{r2bugs.distributions} function to translate parameterizations of common probability distributions from the R to the BUGS implementations.
Although the probability distribution functions are documented in the respective software, we are not aware of any comprehensive treatment of the different parameterizations used by BUGS and R, or a single location in which transformations between these languages are documented.

Parameter transformations, parameter order, and differences in function names are implemented into the R function \code{r2bugs.distributions}. We developed this function to translate the parameterization of common probability distributions from R to BUGS (and back, if \code{direction = 'bugs2r'}.

\newpage
\begin{smallexample}
r2bugs.distributions <- function(priors, direction = 'r2bugs') {
  priors$distn  <- as.character(priors$distn)
  priors$parama <- as.numeric(priors$parama)
  priors$paramb <- as.numeric(priors$paramb)
  ## index dataframe according to distribution
  norm   <- priors$distn %in% c('norm', 'lnorm')    # these have same tramsform
  weib   <- grepl("weib", priors$distn)             # matches r and bugs version
  gamma  <- priors$distn == 'gamma'
  chsq   <- grepl("chisq", priors$distn)            # matches r and bugs version
  bin    <- priors$distn %in% c('binom', 'bin')     # matches r and bugs version
  nbin   <- priors$distn %in% c('nbinom', 'negbin') # matches r and bugs version
  
  ## Normal, log-Normal: Convert sd to precision
  exponent <- ifelse(direction == "r2bugs", -2, -0.5) 
  priors$paramb[norm] <-  priors$paramb[norm] ^ exponent
  
  ## Weibull
  if(direction == 'r2bugs'){
    ## Convert R parameter b to BUGS parameter lambda by l = (1/b)^a
    priors$paramb[weib] <-   (1 / priors$paramb[weib]) ^ priors$parama[weib]
  } else if (direction == 'bugs2r') {
    ## Convert BUGS parameter lambda to BUGS parameter b by b = l^(-1/a)
    priors$paramb[weib] <-  priors$paramb[weib] ^ (- 1 / priors$parama[weib] ) 
  }
  
  ## Reverse parameter order for binomial and negative binomial
  priors[bin | nbin, c('parama', 'paramb')] <-  priors[bin | nbin, c('paramb', 'parama')]
  
  ## Translate distribution names
  if(direction == "r2bugs"){
    priors$distn[weib] <- "weib"
    priors$distn[chsq] <- "chisqr"
    priors$distn[bin]  <- "bin"
    priors$distn[nbin] <- "negbin"
  } else if(direction == "bugs2r"){
    priors$distn[weib] <- "weibull"
    priors$distn[chsq] <- "chisq"
    priors$distn[bin]  <- "binom"
    priors$distn[nbin] <- "nbinom"
  }
  return(priors)
}
\end{smallexample}


As an example, below we state a prior distribution in R $X \sim \mathcal{N}(\mu=10,\sigma=2)$ , convert it to BUGS  $X \sim \mathcal{N}(\mu=10,\tau=1/4)$, and specify a model in JAGS that allows us to sample directly from the prior. The function works for each of the distributions in Table 1. This particular example is equivalent to \code{rnorm(10000, 10, 2)} in R, and is presented as minimal demonstration.

\begin{smallexample}

r.distn <- data.frame(distn = "norm", parama = 10, paramb = 2)
bugs.distn <- r2bugs.distributions(r.distn)

sample.bugs.distn <- function(prior = data.frame(distn = "norm", parama = 0, paramb = 1), n = 10000) {
  require(rjags)
  model.string <- paste0("model{Y ~ d", prior$distn, "(", prior$parama, 
                         ifelse(prior$distn == "chisqr", "", paste0(", ", prior$paramb)), ## chisqr has one param
                         "); a <- x}")  ## tricks JAGS into running without data  
  writeLines(model.string, con = "test.bug")
  j.model  <- jags.model(file = "test.bug", data = list(x = 1))
  mcmc.object <- window(coda.samples(model = j.model, variable.names = c('Y'), n.iter = n * 4, thin = 2),
                        start = n)
  Y <- sample(as.matrix(mcmc.object)[,"Y"], n)
}
X <- sample.bugs.distn(bugs.distn)
\end{smallexample}

\section{Acknowlegements}

This collaboration began on the statistical question and answer website Cross Validated (http://stats.stackexchange.com/q/5543/1381). Funding was provided to DSL by the Energy Biosciences Institute.

\bibliography{lebauer}

\newpage
\address{David S. LeBauer\\
Department of Plant Biology\\
Energy Biosciences Institute\\
University of Illinois, USA}\\
\email{dlebauer@illinois.edu}

\address{Michael C. Dietze\\
Department of Earth And Environment\\
Boston University, USA}

\address{Ben M. Bolker\\
Department of Mathematics and Statistics\\
McMaster University, Canada}
