\title{Translating Probability Distributions:\\ From R to BUGS and Back Again}
\author{by David S. LeBauer, Michael C. Dietze, Ben M. Bolker}

\maketitle

\abstract{
The ability to generate models in the BUGS language simplifies the development and analysis of Bayesian models. 
R provides a powerful interface to software written in the BUGS language, including OpenBUGS, WinBUGS, and JAGS.

However, the R and BUGS languages use inconsistent representations of common probability distributions, and this creates the potential for error and confusion when using both languages. 
Here we review different parameterizations used by the R and BUGS languages, describe how to translate from between the languages, and provide two R functions \code{r2bugs.distributions} and \code{bugs2r.distributions} that transform parameterizations from R to BUGS, and back.
}




Many software packages provide an interface to software written in the BUGS language, enabling users to leverage the power of R when pre-processing data and analyzing model output.
However, many common probability distributions have different default parameterizations in the R versus BUGS languages. 

To support the use of informative prior distributions in meta-analysis \citep{lebauer2012ffb}, we developed functions to translate parameterizations of common probability distributions between the R and BUGS languages. 
Although the probability distribution functions are documented in the respective software, we are not aware of any comprehensive treatment of the different parameterizations used by BUGS and R, or a single location in which transformations between these languages are documented.
Here we provide an overview of distributions for which the default parameterizations used by R and BUGS are different, as well as R functions to translate between R and BUGS.
\renewcommand{\arraystretch}{1.75}
\begin{table}
\begin{center}
\begin{tabular}{llll}
\toprule
 Distribution       &  Language  &  Parameterization  &  Use \\
\toprule
\multirow{2}{*}{Normal}             &  R      &  $\frac{1}{\sqrt{2 \pi}\sigma}\exp(-\frac{(x - \mu)^2}{2 \sigma^2})$          &  \code{dnorm(x, $\mu$, $\sigma$)}   \\
                    &  BUGS   &  $\sqrt{\frac{\tau}{2\pi}}\exp(-(x-\mu)^2\tau)$                                 & \code{dnorm($\mu$, $\tau$)}        \\[1em]
\multirow{2}{*}{log-Normal}         &  R      &  $\frac{1}{\sqrt{2 \pi} \sigma x} \exp(-\frac{(\textrm{log}(x) - \mu)^2}{(2 \sigma^2)})$  &  \code{dlnorm(x, $\mu$, $\sigma$)}  \\
                    &  BUGS   &  $\frac{\sqrt{\tau}}{x}\exp(\frac{-\tau(\textrm{log}(x)-\mu)^2}{2})$                     &  \code{dlnorm($\mu$, $\tau$)}       \\[1em]
 \multirow{2}{*}{Binomial}           &  R      &  ${n \choose x} p^{x}(1-p)^{n-x}$                                               &  \code{dbinom(x, n, p)}             \\
                    &  BUGS   &  same                                                                           &  \code{dbin(p, n)}                  \\[1em]
\multirow{3}{*}{Negative Binomial}  &  R      &  $\frac{\Gamma(x+n)}{\Gamma(n) x!} p^n (1-p)^x$                                 &  \code{dnbinom(x, n, p)}            \\
                    &  R$_{\textrm{alt}}$      &  $\frac{\Gamma(k+x)}{\Gamma(k)x!}(\frac{k}{k+\mu})^k \frac{\mu}{(k+\mu)^x}$     &  \code{dnbinom(x, n, mu = $\mu$)}\\
                    &  BUGS   &  ${x+r-1 \choose x}p^r(1-p)^x$                                                  &  \code{dnegbin(p, r)}               \\[1em]
\multirow{2}{*}{Weibull}            &  R      &  $\frac{a}{b} (\frac{x}{b})^{a-1} \exp(- (\frac{x}{b})^a)$                      &  \code{dweibull(x, a, b)}           \\
                    &  BUGS   &  $\nu\lambda x^{\nu - 1}\exp(-\lambda x^{\nu})$                                 &  \code{dweib($\nu$, $\lambda$)}     \\[1em]
\multirow{3}{*}{Gamma}              &  R      &  ${\frac{r^a}{\Gamma(a)}} x^{a-1} \exp(-xr)$                                    &  \code{dgamma(x, a, r)}             \\
                    &  R$_{\textrm{alt}}$      &  ${\frac{1}{s^{a}\Gamma(a)}} x^{a-1} \exp(-x/s)$                                &  \code{dgamma(x, a, scale = s)}  \\
                    &  BUGS   &  ${\frac{\lambda^r x^{r-1}\exp(-\lambda x)}{\Gamma(r)}}$                        &  \code{dgamma(r, $\lambda$)}        \\
\bottomrule
\end{tabular}
\end{center}
\caption{ Summary of different parameterizations of common distributions used by R and BUGS. The random variable $x$ is implicit in all of the BUGS ``Use'' parameterizations. R$_{\textrm{alt}}$: alternative (non-default) parameterizations in R - these are not used in the \code{r2bugs.distributions} function; to use these parameterizations, the second argument \emph{must} be named. For clarity and ease of reference, parameterizations follow the JAGS and R documentation; thus, the table includes some equivalent variables with different names (e.g. for Gamma,  $r$ in BUGS and $a$ in R are precisely the same), and equivalent expressions with different forms (which motivates this article).}
\label{tab:parameterizations}
\end{table}
\renewcommand{\arraystretch}{1}
\section{Translating R  parameterizations to BUGS (and back again)}

 Translating R paramterizations to BUGS requires three straightforward but error-prone steps.
 The first step is to compare how each language implements the supported probability distribution, in many cases using distinct formulae. 
  Table \ref{tab:parameterizations} summarizes the different parameterizations used in R and the JAGS implementation of BUGS \citep{plummer2011}.
 The second step is to identify relationships between parameters, including transformations and the order of arguments to a function (Table~\ref{tab:transformations}).
 The conversions required to transform parameters from R to BUGS (and BUGS to R) are provided in Table~\ref{tab:transformations}.
 Finally, differences in function names must be addressed (Table~\ref{tab:naming}).

  The \strong{Normal} and \strong{log-normal} distributions use standard deviation ($\sigma$) in R and precision ($\tau=1/\sigma^2$) in BUGS; mean $\mu$ is the first argument in both languages.
  The \strong{Negative binomial} distribution uses a continuous size parameter ($n$) in R and a discrete size parameter ($r$) in BUGS; both languages use a probability parameter ($p$) by default, but the order of parameters is reversed. 
  The R functions \code{*nbinom} can also use $\mu$ (where $p=n/(n+\mu)$).
  The \strong{Weibull} distribution has shape ($a$) and scale ($b$) parameters in R, but has shape ($\nu$) and lambda ($\lambda$, where $\lambda=1/b$) in BUGS.
  The default parameterization of the \strong{Gamma} in R uses shape ($a$) and rate ($r$) whereas BUGS reverses and renames these parameters: rate ($r$) and shape ($\lambda$).
  R also allows the Gamma to accept shape and scale parameters, if the scale argument is explicitly named (e.g. \code{dgamma(x, a, scale = b)}).
  The \strong{Beta}, \strong{Poisson}, \strong{Exponential}, and \strong{Uniform} distributions have the same parameterizations in both BUGS and R.

  The order of parameters matters, since argument names are not used in BUGS and are optional in R.
  This is especially important because the order of parameters for the \strong{Binomial} and \strong{Negative Binomial} distributions are switched when translating between BUGS and R (Tables~\ref{tab:parameterizations},\ref{tab:transformations}).
  Alternative parameterizations of the \strong{Gamma} and \strong{Negative Binomial} distributions are provided in R, but their use requires that the parameters be given as named arguments.

 All of these considerations are integrated into the R functions \code{r2bugs.distributions} and \\ \code{bugs2r.distributions} that translate the parameterization of common probability distributions between the R and BUGS languages (see Appendix).

\begin{table}
\begin{center}
\begin{tabular}{ll}
\toprule
 Distribution        &  R to BUGS conversion                      \\
\midrule
 Normal, log-Normal  &  $\tau = 1/\sigma^2$                   \\
 Binomial            &  reverse parameter order                \\
 Negative Binomial   &  reverse parameter order \\
 Weibull             &  $\lambda = (1/b)^{a}$  \\
 Gamma               &  $r = a$; $\lambda = r$ (reverse order)                           \\
\bottomrule
\end{tabular}
\end{center}
\caption{Transformations required to convert from R to BUGS parameterizations}
\label{tab:transformations}
\end{table}

In addition to different parameterizations, four distributions have different naming conventions (Table~\ref{tab:naming}).

\begin{table}
\begin{center}
\begin{tabular}{lll}
\toprule
 Distribution       &  R         &  BUGS     \\
\midrule
 Binomial           &  \code{dbinom}    &  \code{dbin}     \\
 Negative Binomial  &  \code{dnbinom}   &  \code{dnegbin}  \\
 $\chi$$^2$         &  \code{dchisq}    &  \code{dchisqr}  \\
 Weibull            &  \code{dweibull}  &  \code{dweib}    \\
\bottomrule
\end{tabular}
\end{center}
\caption{Distributions with different naming conventions}
\label{tab:naming}
\end{table}

\section{Acknowlegements}

This collaboration began on the statistical question and answer website Cross Validated \\(\url{http://stats.stackexchange.com/q/5543/1381}). Funding was provided to DSL by the Energy Biosciences Institute.

\bibliography{lebauer}

\newpage
\address{David S. LeBauer\\
  Department of Plant Biology\\
  Energy Biosciences Institute\\
  University of Illinois, USA}\\
\email{dlebauer@illinois.edu}

\address{Michael C. Dietze\\
  Department of Earth And Environment\\
  Boston University, USA}

\address{Ben M. Bolker\\
  Department of Mathematics and Statistics\\
  McMaster University, Canada}


\section{Appendix}

\subsection{R functions r2bugs and bugs2r to translate between R and BUGS parameterizations of common probability distributions}

\begin{example}
r2bugs.distributions <- function(priors, direction = 'r2bugs') {

  priors$distn  <- as.character(priors$distn)
  priors$parama <- as.numeric(priors$parama)
  priors$paramb <- as.numeric(priors$paramb)

  ## index dataframe according to distribution
  norm   <- priors$distn %in% c('norm', 'lnorm')    # these have same tramsform
  weib   <- grepl("weib", priors$distn)             # matches r and bugs version
  gamma  <- priors$distn == 'gamma'
  chsq   <- grepl("chisq", priors$distn)            # matches r and bugs version
  bin    <- priors$distn %in% c('binom', 'bin')     # matches r and bugs version
  nbin   <- priors$distn %in% c('nbinom', 'negbin') # matches r and bugs version

  ## Check that no rows are categorized into two distributions
  if(max(rowSums(cbind(norm, weib, gamma, chsq, bin, nbin))) > 1) {
    badrow <- rowSums(cbind(norm, weib, gamma, chsq, bin, nbin)) > 1
    stop(paste(unique(priors$distn[badrow])),
         "are identified as > 1 distribution")
  }

  exponent <- ifelse(direction == "r2bugs", -2, -0.5) 
  ## Convert sd to precision for norm & lnorm
  priors$paramb[norm] <-  priors$paramb[norm] ^ exponent
  if(direction == 'r2bugs'){
    ## Convert R parameter b to BUGS parameter lambda by l = (1/b)^a
    priors$paramb[weib] <-   (1 / priors$paramb[weib]) ^ priors$parama[weib]
  } else if (direction == 'bugs2r') {
    ## Convert BUGS parameter lambda to BUGS parameter b by b = l^(-1/a)
    priors$paramb[weib] <-  priors$paramb[weib] ^ (- 1 / priors$parama[weib] )
 
  }
  ## Reverse parameter order for binomial and negative binomial
  priors[bin | nbin, c('parama', 'paramb')] <-  priors[bin | nbin, c('paramb', 'parama')]
  
  ## Translate distribution names
  if(direction == "r2bugs"){
    priors$distn[weib] <- "weib"
    priors$distn[chsq] <- "chisqr"
    priors$distn[bin]  <- "bin"
    priors$distn[nbin] <- "negbin"
  } else if(direction == "bugs2r"){
    priors$distn[weib] <- "weibull"
    priors$distn[chsq] <- "chisq"
    priors$distn[bin]  <- "binom"
    priors$distn[nbin] <- "nbinom"
  }
  return(priors)
}

bugs2r.distributions <- function(..., direction = "bugs2r") {
  return(r2bugs.distributions(..., direction))
}

\end{example}
