% !TeX root = RJwrapper.tex
\title{Translating Probability Density Functions: From R to BUGS and Back Again}
\author{by David S.\ LeBauer, Michael C.\ Dietze, Benjamin M.\ Bolker}

\maketitle

\abstract{The ability to implement statistical models in the BUGS language facilitates Bayesian inference by automating MCMC algorithms.
Software packages that interpret the BUGS language include OpenBUGS, WinBUGS, and JAGS.
R packages that link BUGS software to the R environment, including \CRANpkg{rjags} and \CRANpkg{R2WinBUGS}, are widely used in Bayesian analysis. 
Indeed, many packages in the Bayesian task view on CRAN (\url{http://cran.r-project.org/web/views/Bayesian.html}) depend on this integration. However, the R and BUGS languages use different representations of common probability density functions,  creating a potential for errors to occur in the implementation or interpretation of analyses that use both languages.
Here we review different parameterizations used by the R and BUGS languages, describe how to translate between the languages, and provide an R function, \code{r2bugs.distributions}, that transforms parameterizations from R to BUGS and back again.
}

\begin{widetable}[ht!]

{\small
\begin{tabular}{llllll}
\toprule
Distribution       &  Lang.  &  Parameterization  &  Use &  Notes \\
\midrule
Normal&&&&\\
&  R      &  $\frac{1}{\sqrt{2 \pi}\sigma}\exp\left(-\frac{\left(x - \mu\right)^2}{2 \sigma^2}\right)$ & \code{dnorm($x$, mean = $\mu$, sd = $\sigma$)}& \\
&  BUGS   &  $\sqrt{\frac{\tau}{2\pi}}\exp\left(-\left(x-\mu\right)^2\tau\right)$ & \code{dnorm(mean = $\mu$, precision = $\tau$)} & $\tau=\left(\frac{1}{\sigma}\right)^2$  \\
\midrule
log-Normal&&&&\\
&  R      &  $\frac{1}{\sqrt{2 \pi} \sigma x} \exp\left(-\frac{\left(\textrm{log}\left(x\right) - \mu\right)^2}{\left(2 \sigma^2\right)}\right)$  &  \code{dlnorm($x$, mean = $\mu$, sd = $\sigma$)}  & \\
&  BUGS   &  $\frac{\sqrt{\tau}}{x}\exp\left(\frac{-\tau\left(\textrm{log}\left(x\right)-\mu\right)^2}{2}\right)$                     &  \code{dlnorm(mean = $\mu$, precision = $\tau$)} & $\tau=\left(\frac{1}{\sigma}\right)^2$\\
\midrule
Binomial&&&&  reverse parameter order\\
&  R      &  ${n \choose x} p^{x}\left(1-p\right)^{n-x}$                                               &  \code{dbinom($x$, size = $n$, prob = $p$)}  &  \\
&  BUGS   &  same   &  \code{dbin(prob = $p$, size = $n$)} & \\
\midrule
Negative Binomial&&&&reverse parameter order\\
&  R      &  $\frac{\Gamma\left(x+n\right)}{\Gamma\left(n\right) x!} p^n \left(1-p\right)^x$ &  \code{dnbinom($x$, size = $n$, prob = $p$)} & size (\code{n}) is continuous    &     \\
%&  R$^*$      &  $\frac{\Gamma(k+x)}{\Gamma(k)x!}(\frac{k}{k+\mu})^k \frac{\mu}{(k+\mu)^x}$     &  \code{dnbinom(x, size = $n$, mu = $\mu$)} & \\%$\mu=n(1-p)/p$  \\
&  BUGS   &  ${x+r-1 \choose x}p^r\left(1-p\right)^x$ &  \code{dnegbin(prob = $p$, size = $r$)}     & size (\code{r}) is discrete  \\ 
\midrule
Weibull&&&&\\
&  R      &  $\frac{a}{b} (\frac{x}{b})^{a-1} \exp\left(- \left(\frac{x}{b}\right)^a\right)$                      &  \code{dweibull($x$, shape = $a$, scale = $b$)}          & \\
&  BUGS   &  $\nu\lambda x^{\nu - 1}\exp\left(-\lambda x^{\nu}\right)$  &  \code{dweib(shape = $\nu$, lambda = $\lambda$)}   & $\lambda=\left(\frac{1}{b}\right)^a$  \\
\midrule
Gamma&&&&reverse parameter order \\
&  R      &  ${\frac{r^a}{\Gamma(a)}} x^{a-1} \exp(-xr)$                                    &  \code{dgamma($x$, shape = $$a, rate = $r$)} & \\
%&  R$^*$      &  ${\frac{1}{s^{a}\Gamma(a)}} x^{a-1} \exp(-x/s)$                                &  \code{dgamma(x, shape = a, scale = s)} &  \\
&  BUGS   &  ${\frac{\lambda^r x^{r-1}\exp(-\lambda x)}{\Gamma(r)}}$                        &  \code{dgamma(shape = $r$, lambda = $\lambda$)}  &  \\
\bottomrule
\end{tabular}
}

\caption{Summary of different parameterizations of common
distributions used by R and BUGS. \strong{Note:} For ease of reference, parameterizations follow the JAGS and R documentation;
as a result, the table includes equivalent equations that appear different, either because JAGS and R use different names for the same parameter or because the equation has been rearranged. For example, the shape parameter of the \emph{Gamma} distribution is $r$ in the BUGS documentation and $a$ in the R documentation.
For the \emph{Binomial}, \emph{Negative Binomial}, and \emph{Gamma} distributions, BUGS and R expect parameters in different order (the order of parameters matters since arguments are assigned based on position in BUGS and may be in R as well).
R allows alternate parameterizations for the \emph{Negative Binomial} and \emph{Gamma} distributions, but these are not shown here.
The variable $x$ is implicit in all of the BUGS ``Use'' expressions.
The \emph{Beta}, \emph{Poisson}, \emph{Exponential}, and \emph{Uniform} distributions have identical parameterizations in R and BUGS.
}
\label{tab:parameterizations}
\end{widetable}


\section{Probability density functions in R and BUGS}

R and BUGS implement many of the same probability distribution functions,
but they often parameterize the same distribution differently
(Table~\ref{tab:parameterizations}).
Although these probability distribution functions 
are clearly described in the documentation of their respective languages, 
we were unable to find a summary of these differences in one place.
The motivation for this article is to document and clarify these differences. 
Our sources are the JAGS documentation \citep{plummer2011} 
and the documentation of individual R functions.

\section{A bilingual translation function}

To support the automation of model specification in JAGS with priors computed and stored in R \citep{lebauer2012ffb}, we developed a function to translate parameterizations of common probability distributions from R to BUGS (and back again, by specifying \code{direction = 'bugs2r'}).
Parameter transformations, parameter order, and differences in function names are documented in Table~\ref{tab:parameterizations} and implemented in the R function \code{r2bugs.distributions}. 

\begin{example}
r2bugs.distributions <- function(priors, direction = 'r2bugs') {
  priors$distn  <- as.character(priors$distn)
  priors$parama <- as.numeric(priors$parama)
  priors$paramb <- as.numeric(priors$paramb)
  ## index dataframe according to distribution
  norm   <- priors$distn %in% c('norm', 'lnorm')    # these have same transform
  weib   <- grepl("weib", priors$distn)             # matches r and bugs version
  gamma  <- priors$distn == 'gamma'
  chsq   <- grepl("chisq", priors$distn)            # matches r and bugs version
  bin    <- priors$distn %in% c('binom', 'bin')     # matches r and bugs version
  nbin   <- priors$distn %in% c('nbinom', 'negbin') # matches r and bugs version
  
  ## Normal, log-Normal: Convert sd to precision
  exponent <- ifelse(direction == "r2bugs", -2, -0.5) 
  priors$paramb[norm] <-  priors$paramb[norm] ^ exponent
  
  ## Weibull
  if(direction == 'r2bugs'){
    ## Convert R parameter b to BUGS parameter lambda by l = (1/b)^a
    priors$paramb[weib] <- (1 / priors$paramb[weib]) ^ priors$parama[weib]
  } else if (direction == 'bugs2r') {
    ## Convert BUGS parameter lambda to BUGS parameter b by b = l^(-1/a)
    priors$paramb[weib] <-  priors$paramb[weib] ^ (- 1 / priors$parama[weib] ) 
  }
  
  ## Reverse parameter order for binomial and negative binomial
  priors[bin | nbin, c('parama', 'paramb')] <-
    priors[bin | nbin, c('paramb', 'parama')]
  
  ## Translate distribution names
  if(direction == "r2bugs"){
    priors$distn[weib] <- "weib"
    priors$distn[chsq] <- "chisqr"
    priors$distn[bin]  <- "bin"
    priors$distn[nbin] <- "negbin"
  } else if(direction == "bugs2r"){
    priors$distn[weib] <- "weibull"
    priors$distn[chsq] <- "chisq"
    priors$distn[bin]  <- "binom"
    priors$distn[nbin] <- "nbinom"
  }
  return(priors)
}
\end{example}

\section{A simple example}

As an example, we take the R-parameterized prior distribution $X \sim \mathcal{N}(\mu=10,\sigma=2)$ and 
convert it to BUGS  parameterization $X \sim \mathcal{N}(\mu=10,\tau=1/4)$.
We specify a model in JAGS that allows us to sample directly from a prior distribution.  
The function works for each of the distributions in Table~\ref{tab:parameterizations}. 
This particular example is the JAGS implementation of \samp{rnorm(10000, 10, 2)} in R. 
It is presented as minimal demonstration;
for a non-trivial application, see \cite{lebauer2012ffb}. 

\begin{example}

r.distn <- data.frame(distn = "norm", parama = 10, paramb = 2)
bugs.distn <- r2bugs.distributions(r.distn)
   
sample.bugs.distn <- function(prior = data.frame(distn = "norm", parama = 0, 
                                paramb = 1), n = 10000) {
  require(rjags)
  model.string <- paste0(
    "model{Y ~ d", prior$distn, 
    "(", prior$parama, 
    ## chisqr has only one parameter
    ifelse(prior$distn == "chisqr", "", paste0(", ", prior$paramb)), ");",
    "a <- x}"   ## trick JAGS into running without data 
  )    
  writeLines(model.string, con = "test.bug")
  j.model  <- jags.model(file = "test.bug", data = list(x = 1))
  mcmc.object <- window(
    coda.samples(
      model = j.model, variable.names = c('Y'), 
      n.iter = n * 4, thin = 2),
    start = n)
  Y <- sample(as.matrix(mcmc.object)[,"Y"], n)
}
X <- sample.bugs.distn(bugs.distn)
\end{example}

\section{Acknowlegements}

This collaboration began on the Cross Validated statistical forum (\url{http://stats.stackexchange.com/q/5543/1381}). Funding was provided to DSL and MCD by the Energy Biosciences Institute.

\bibliography{lebauer}

\begin{multicols}{2}

\address{David S.\ LeBauer\\
Energy Biosciences Institute\\
University of Illinois\\
USA}\\
\email{dlebauer@illinois.edu}

\address{Michael C.\ Dietze\\
Department of Earth And Environment\\
Boston University\\
USA}
%\\ \email{dietze@bu.edu}
\columnbreak

\address{Benjamin M.\ Bolker\\
Department of Mathematics \& Statistics\\
McMaster University\\
Canada}
%\\ \email{bolker@mcmaster.ca}

\end{multicols}
