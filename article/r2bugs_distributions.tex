\documentclass[a4paper]{report}
\usepackage{RJournal}
\usepackage[round]{natbib}
\bibliographystyle{plainnat}

%% load any required packages here

\begin{document}
\SweaveOpts{concordance=TRUE}

%% do not edit, for illustration only
\fancyhf{}
\fancyhead[LO,RE]{\textsc{Contributed Article}}
\fancyhead[RO,LE]{\thepage}
\fancyfoot[L]{The R Journal Vol. X/Y, Month, Year}
\fancyfoot[R]{ISSN 2073-4859}

%% replace RJtemplate with your article
\begin{article}

\title{A review of different parameterizations used in R and BUGS}
\author{by David LeBauer, Ben Bolker, Michael C. Dietze}

\maketitle

\abstract{
 The ability to generate models in the BUGS simplifies the development and analysis of Bayesian models.
 Software written in the BUGS language includes OpenBUGS, WinBUGS, and JAGS.
 A number of packages provide an access to BUGS based software, enabling users to leverage the power of R when pre-processing data and analyzing model output.

 However, many common probability distributions have different default parameterizations in the R versus BUGS languages.
 As part of a semi-automated meta-analysis module in the PEcAn workflow \citep{lebauer2012ffb}, we developed a function to transform parameterizations between the R and BUGS languages. 
 Although the probability distribution functions are documented in the respective software, we are not aware of any comprehensive treatment of the different parameterizations used by BUGS and R, or a single location in which transformations between these languages are documented.
 Here we provide an overview of distributions for which the default parameterizations used by R and BUGS are different, including the transformations required to convert from one language to another and a R function that encapsulates these transformations.
}

\section{A comparison of parameterizations}

Table \ref{tab:parameterizations} summarizes the different parameterizations used in R \citep{rdev2011} and BUGS \citep{plummer2011}.  

{\tiny
\begin{table}\label{tab:parameterizations}
\caption{Summary of different parameterizations of common distributions used by R and BUGS. The random variable $x$ is implicit in all of the BUGS ``Use'' parameterizations. $^*$ non-default parameterizations in R - these are not used in the \code{r2bugs.distributions} function; to use these parameterizations, the second argument \emph{must} be named. For clarity and ease of reference, parameterizations follow the JAGS and R documentation; thus, the table includes some equivalent variables with different names (e.g. for Gamma,  $r$ BUGS and $a$ in R are precisely the same), and equivalent expressions with different forms (which motivates this article).}
\begin{center}
\begin{tabular}{llll}
\hline
 Distribution       &  Language  &  Parameterization  &  Use \\
\hline
 Normal             &  R      &  $\frac{1}{\sqrt{2 \pi}\sigma}\exp(-\frac{(x - \mu)^2}{2 \sigma^2})$          &  \code{dnorm(x, $\mu$, $\sigma$)}   \\
                    &  BUGS   &  $\sqrt{\frac{\tau}{2\pi}}\exp(-(x-\mu)^2\tau)$                                 & \code{dnorm($\mu$, $\tau$)}        \\
 log-Normal         &  R      &  $\frac{1}{\sqrt{2 \pi} \sigma x} \exp(-\frac{(log x - \mu)^2}{(2 \sigma^2)})$  &  \code{dlnorm(x, $\mu$, $\sigma$)}  \\
                    &  BUGS   &  $\frac{\sqrt{\tau}}{x}\exp(\frac{-\tau(log(x)-\mu)^2}{2})$                     &  \code{dlnorm($\mu$, $\tau$)}       \\
 Binomial           &  R      &  ${n \choose x} p^{x}(1-p)^{n-x}$                                               &  \code{dbinom(x, n, p)}             \\
                    &  BUGS   &  same                                                                           &  \code{dbin(p, n)}                  \\
 Negative Binomial  &  R      &  $\frac{\Gamma(x+n)}{\Gamma(n) x!} p^n (1-p)^x$                                 &  \code{dnbinom(x, n, p)}            \\
                    &  R$^*$      &  $\frac{\Gamma(k+x)}{\Gamma(k)x!}(\frac{k}{k+\mu})^k \frac{\mu}{(k+\mu)^x}$     &  \code{dnbinom(x, n, mu = $\mu$)}$^*$        \\
                    &  BUGS   &  ${x+r-1 \choose x}p^r(1-p)^x$                                                  &  \code{dnegbin(p, r)}               \\
 Weibull            &  R      &  $\frac{a}{b} (\frac{x}{b})^{a-1} \exp(- (\frac{x}{b})^a)$                      &  \code{dweibull(x, a, b)}           \\
                    &  BUGS   &  $\nu\lambda x^{\nu - 1}\exp(-\lambda x^{\nu})$                                 &  \code{dweib($\nu$, $\lambda$)}     \\
 Gamma              &  R      &  ${\frac{r^a}{\Gamma(a)}} x^{a-1} \exp(-xr)$                                    &  \code{dgamma(x, a, r)}             \\
                    &  R$^*$      &  ${\frac{1}{s^{a}\Gamma(a)}} x^{a-1} \exp(-x/s)$                                &  \code{dgamma(x, a, scale = s)}$^*$  \\
                    &  BUGS   &  ${\frac{\lambda^r x^{r-1}\exp(-\lambda x)}{\Gamma(r)}}$                        &  \code{dgamma(r, $\lambda$)}        \\
\hline
\end{tabular}
\end{center}
\end{table}
}

\section{Converting from R to BUGS parameterizations}

  The Normal and log-normal are parameterized in terms of $\tau$ (precision) by BUGS and in terms of $\sigma$ (standard deviation) or $\sigma$$^2$ (variance) by R.
  The Negative binomial distribution size parameter is discrete in BUGS (size, prob), but the size parameter is continuous in R (size, mu).
  The Weibull distribution has parameters ($\nu=\textrm{shape}, \lambda= \textrm{lambda}$) in BUGS, in R, it has either (a=shape, b=scale) or (a=shape,b=rate).
  The default parameterization of the Gamma in R is ($a$=shape, $r$ = rate) whereas in BUGS the parameterization is ($r$ = rate, $\lambda$ = shape). 
  Thus, the order of the shape and rate parameters is reversed in BUGS (which uses rate, shape).
  R also allows the Gamma to accept (shape, scale) but this requires that the scale argument is named (e.g. \code{dgamma(x, a, scale = b)}).

  The Beta, Poisson, Exponential, and Uniform distributions have the same parameterizations in both BUGS and R.
  A summary of these distributions is provided in table 1; the conversions required to transform parameters from R to BUGS (and BUGS to R) is provided in the table 2 and the functions \code{r2bugs} and \code{bugs2r} in the appendix.

  It is also important to note that order matters, particularly in BUGS which doesn't have named arguments. 
  For example, the R prarmeterization of the Binomial distribution, \code{dbinom(x, size, prob)}, has the same parameters but they are in reversed order in the BUGS distribution, \code{dbin(p,n)}.

%\begin{table}\label{tab:transformations}
\small{\begin{center}
\begin{tabular}{ll}
\hline
 Distribution        &  R to BUGS conversion                      \\
\hline
 Normal, log-Normal  &  $\tau = 1/\sigma^2$                   \\
 Binomial            &  reverse parameter order                \\
 Negative Binomial   &  reverse parameter order \\
 Weibull             &  $\lambda = (1/b)^{a}$  \\
 Gamma               &  $r = a$; $\lambda = r$ (reverse order)                           \\
\hline
\end{tabular}
\end{center}}
%\end{table}
 
\section{Differences in naming}

In addition to different parameterizations, four distributions have different naming conventions.

%\begin{table}\label{tab:naming}
\begin{center}
\begin{tabular}{lll}
\hline
 Distribution       &  R         &  BUGS     \\
\hline
 Binomial           &  \code{dbinom}    &  \code{dbin}     \\
 Negative Binomial  &  \code{dnbinom}   &  \code{dnegbin}  \\
 $\chi$$^2$         &  \code{dchisq}    &  \code{dchisqr}  \\
 Weibull            &  \code{dweibull}  &  \code{dweib}    \\
\hline
\end{tabular}
\end{center}
%\end{table}

\section{Implementation}

As a simple example, we will compare the use of R and BUGS to sample from a normal distribution, $N(\mu = 10, \sigma = 2)$
{\scriptsize








